\subsection{Mission Statement}
The mission of this project is to propose a Low Earth Orbit, 6U sized Cubesat to monitor and track down hurricanes across the South-Atlantic U.S. Region. This Cubesat should be able to measure small to big scale hurricanes across the required region with greater accuracy. It should also be able to record the size of a hurricane, speed, impact area,  changes in temperature, pressure, and other details successfully. By doing so, it has to communicate and transmit data back and forth between local stations effectively. Furthermore, it will ensure the collected data is processed and used as per the contract agreement with the NASA and its interested parties effectively.
\subsection{Mission Relevance to NASA}
HurriSat's mission is in accordance with the first objective of NASA's Strategic plan  \cite{NASA2018} listed in sections 1.1, 1.2, and 3. It's main goal is to study the causes & effects of severe space weather events, and prevent potential damage. Additional criteria in addressing the National Challenges and Catalyze economic growth for space sustainability. Climate change greatly increases the severity and frequency of hurricanes and other extreme weather. This has led to a significant need for fast and accurate weather data in order to prevent and reduce catastrophic damages. This feasibility report ensures HurriSat meets the requirements and objective outlined by NASA in the CSLI \cite{CLSI} initiative.

This document provides a refined look at the mission analysis and concept of operations for HurriSat. The mission goals and objectives are laid out, and the analysis plan is discussed in detail. The conceptual design and analysis takes a scientific approach towards the concept of operations. Here the criteria from the NASA CLSI missions are analyzed and turned into either requirements or constraints. Alternate plans for the mission are also discussed in later section. Furthermore, a brief evaluation is conducted for each subsystem and its purpose in the mission according to the project plan.

\subsection{Background in Hurricanes}
    \begin{wraptable}{r}{0.5\textwidth}
    \centering
    \begin{tabular}{ccr}
    \rowcolor{gray!50}{Category & Speed (mph) & Severity}\\
    \cellcolor{red!20} 1  & { $74-95$} & {\small Minimal damage}\\
    \cellcolor{red!40} 2  & { $96-110$} & {\small Considerable damage}\\
    \cellcolor{red!60} 3 & { $111-129$} & {\small Extreme damage}\\
    \cellcolor{red!80} 4 & { $130-156$} & {\small Devastating}\\
    \cellcolor{red} 5 & { $>156$} & {\small Catastrophic}\\
    \end{tabular}
    \caption{\small Hurricane Severity}
    \label{Tab:hss}
\end{wraptable} 
    Hurricanes are among the most destructive weather phenomena that are caused by strong winds and surge effects. They can be fatal, and usually cause destruction of infrastructures. In 2012, Hurricane Sandy led to about $\$65$ billion damage in the northeastern coastal region of the USA and in the Ontario Province of Canada \cite{Cui2016}. The severity of the consequences, associates with hurricane activity, has motivated several investigations, attempting to forecast hurricane activity in both near and distant future. Hurricanes are primarily classified based on their speed as seen in Table \ref{Tab:hss}, with Category 5 being the most severe. This serves as one of the many reasons that led us bring forward project HurriSat to NASA and its affiliates. Therefore, HurriSat's sole mission will be dedicated to tracking, monitoring and possibly avoiding fatal devastation caused by hurricanes. Additionally, HurriSat will be equipped with high-tech cameras and faster processor to provide accurate weather data with little to zero maintenance cost.

\subsection{Stakeholders and Customers} The stake holders partly consist of with an interest with the enterprise  of NASA and its affiliates. They are divided into two main categories, Primary and Secondary stakeholders as listed in Table \ref{Tab:st}. Those stakeholders may include those listed but not limited to:
\begin{atn}[H]{Stakeholders}{tab:sh}{l l @{\hskip 25pt} r r}
    \textcolor{black}{Primary Stakeholders} & Secondary Stakeholders \\
    NASA & Department of Education\\
    National Weather Service & National Environmental Satellit\\
    Federal Emergency Management Agency &  NOAA \\
    Department of Defense & Others: Public Safety, Health and Red cross \\
\label{Tab:sh}
\end{atn}   
